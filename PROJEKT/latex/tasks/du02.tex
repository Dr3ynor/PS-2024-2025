\section*{Úkol 2}
\label{sec:task-2}

Porovnejte nárůsty ve výkonnostních skórech (FPS) pro verzi hry "Cyberpunk 2077" po aplikaci 1.5 patche (dále jen \textbf{„nárůst FPS“}) 
pro vybrané grafické karty. Nezapomeňte, že použité metody mohou vyžadovat splnění určitých předpokladů. Pokud tomu tak bude, okomentujte 
splnění/nesplnění těchto předpokladů jak na základě explorační analýzy (např. s odkazem na histogram apod.), tak exaktně pomocí metod statistické indukce.

\begin{enumerate}[label=\alph*]
    \item Graficky prezentujte srovnání nárůstu FPS pro grafické karty Nvidia RTX 3070 Ti a AMD Radeon RX 7700 XT (vícenásobný krabicový graf, histogramy, q-q grafy). 
    Srovnání okomentujte (včetně informace o případné manipulaci s datovým souborem). 
    \textbf{Poznámka:} \textit{Byla-li grafická prezentace FPS v úkolu 1 bez připomínek, stačí do komentáře vložit odkaz na grafické výstupy z úkolu 1.}

    \newpage
    \item Na hladině významnosti 5 \% rozhodněte, zda jsou střední hodnoty nárůstů FPS (popř. mediány nárustů FPS) pro grafické karty Nvidia RTX 3070 Ti a 
    AMD Radeon RX 7700 XT statisticky významné. K řešení využijte bodové a intervalové odhady i čistý test významnosti. Výsledky okomentujte.

    \newpage
    \item Pro grafické karty Nvidia RTX 3070 Ti a AMD Radeon RX 7700 XT rozhodněte (na hladině významnosti 5 \%), zda se jejich střední hodnoty (popř. mediány) 
    nárůstu FPS po aplikaci 1.5 patche statisticky významně liší. K řešení využijte bodový a intervalový odhad i čistý test významnosti. Výsledky okomentujte.
\end{enumerate}

\endinput